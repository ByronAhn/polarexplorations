\tikzset{
    dots/.style={
        line width=4pt,
        line cap=round,
        dash pattern=on 0pt off 6pt
    }
}
\begin{tikzpicture}[scale=1.25]
%\node[anchor=north west, xscale=.225, yscale=.225] at (-0.525,4.15) {\includegraphics{ranges-what-if4.png}};
\node[rotate=90] at (-0.75,3) {\relsize{-2}f0 (Hz)};
\begin{scope}[anchor=east, style={font=\relsize{-2}}]
%	\node at (0,0) {0};
%	\node at (0,1) {100};
	\node at (0,2) {200};
	\node at (0,3) {300};
	\node at (0,4) {400};
\end{scope}
\foreach \y in {2,...,4}
{
	\draw (0,\y) -- (-.1,\y);
}
\begin{scope}
\draw (0,4.25) rectangle (6.4,1.75);
\draw (0.0,1.75) rectangle (1.6,1.25);
\node at (.8, 1.5) {\relsize{-2}word1};
\draw (1.6,1.75) rectangle (3.2,1.25);
\node at (2.4, 1.5) {\relsize{-2}word2};
\draw (3.2,1.75) rectangle (4.8,1.25);
\node at (4, 1.5) {\relsize{-2}word3};
\draw (4.8,1.75) rectangle (6.4,1.25);
\node at (5.6, 1.5) {\relsize{-2}word4};
%
\draw (0,1.25) rectangle (6.4,0.75);
\draw (1.3, 1.25) -- (1.3, 0.75);
\node[fill=white] at (1.3, 1) {\relsize{-2}H*};
\draw (2.8, 1.25) -- (2.8, 0.75);
\node[fill=white] at (2.8, 1) {\relsize{-2}H*};
\draw (6.05, 1.25) -- (6.05, 0.75);
\node[fill=white] at (6.05, 1) {\relsize{-2}L-L\%};
%
\coordinate (pt1) at (0.2,2.85);
\coordinate (pt2) at (1.4,3.95);
\coordinate (pt2a) at (2.1,3.45);
\coordinate (pt2b) at (2.1,3);
\coordinate (pt3) at (2.8,3.95);
\coordinate (pt4) at (3.95,2.25);
\coordinate (pt5) at (6.2,2.25);
%\foreach \x in {1,...,14}
%{
%	\draw[fill] (pt\x) circle (.05);
%}
\draw[CB1, dots] (pt1) sin (pt2) -- (pt3) cos ([xshift=0,yshift=7]pt4) sin ([xshift=7,yshift=0]pt4) -- (pt5);
\draw[CB3, line width=2.33pt, line cap=round, dashed] (pt1) sin (pt2) cos ([xshift=-5,yshift=2]pt2a) sin (pt2a) cos ([xshift=5,yshift=2]pt2a) sin (pt3) cos ([xshift=0,yshift=7]pt4) sin ([xshift=7,yshift=0]pt4) -- (pt5);
\draw[CB4, line width=2pt, dotted] (pt1) sin (pt2) cos ([xshift=-5,yshift=2]pt2b) sin (pt2b) cos ([xshift=5,yshift=2]pt2b) sin (pt3) cos ([xshift=0,yshift=7]pt4) sin ([xshift=7,yshift=0]pt4) -- (pt5);
\end{scope}
\end{tikzpicture}