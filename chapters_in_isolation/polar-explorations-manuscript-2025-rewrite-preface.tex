\documentclass[11pt, twoside]{memoir}

\setsecnumdepth{subsubsection}
\settocdepth{subsubsection}

\usepackage[normalem]{ulem}

\usepackage{fontspec}
	\setmainfont%
		[Mapping=tex-text]
	{Cambria}
	\setsansfont%
%	[Ligatures={NoCommon, NoDiscretionary},%
		[Mapping=tex-text]%
		{Inconsolata}
\usepackage{relsize}

\def\textlabel#1{{\relsize{-.5}\fontspec[Mapping=tex-text]{Roboto Mono}{#1}}}
\def\langtext#1{\textit{#1}}

\usepackage{float}

\usepackage[top=1in, bottom=1in, left=1.25in, right=1.25in]{geometry}

\usepackage{xcolor}

\usepackage{hyperref}

\renewcommand\cftappendixname{\appendixname~}
\usepackage[nonumberlist, nopostdot, section=chapter, numberedsection=autolabel]{glossaries}
\makeglossaries

\usepackage{tikz}
\usetikzlibrary{shapes, arrows, backgrounds, fit, positioning}
\usetikzlibrary{decorations.pathreplacing}
\tikzset{
    dots/.style={
        line width=4pt,
        line cap=round,
        dash pattern=on 0pt off 6pt
    }
}
\usepackage{colortbl}
	\definecolor{CB1}{RGB}{0, 114, 178}%BLUE
	\definecolor{CB2}{RGB}{213, 94, 0}%VERMILLION
	\definecolor{CB3}{RGB}{0, 158, 115}%BLUE GREEN
	\definecolor{CB4}{RGB}{204, 121, 167}%REDDISH PURPLE
	\definecolor{CB5}{RGB}{86, 180, 233}%SKY BLUE
	\definecolor{CB6}{RGB}{230, 159, 0}%ORANGE
	\definecolor{CB7}{RGB}{240, 228, 66}%YELLOW
	\def\CB#1#2{\textcolor{CB#1}{#2}}
\usepackage{booktabs, longtable, array, arydshln, multirow}
\usepackage{caption}
\newlength\defaultaboverulesep
\setlength\defaultaboverulesep{\aboverulesep}
\newlength\defaultbelowrulesep
\setlength\defaultbelowrulesep{\belowrulesep}
\setlength\aboverulesep{0pt}
\setlength\belowrulesep{0pt}
\def\arraystretch{1.15}

\usepackage[framemethod=tikz]{mdframed}
	\newmdenv[middlelinecolor=CB1,
		middlelinewidth=1pt,
		backgroundcolor=gray!50,
		roundcorner=4pt]{infobox}
	\newmdenv[middlelinecolor=CB2,
		middlelinewidth=2pt,
		backgroundcolor=CB2!30,
		roundcorner=4pt]{toexpand}

\usepackage{makeidx}
\usepackage{graphicx}
	\graphicspath{{figures}}

\usepackage{expex}

\usepackage{enumitem}
	\setitemize{itemsep=.5ex, topsep=1ex, parsep=0pt, partopsep=0pt, leftmargin=3em, rightmargin=0ex}
	\setenumerate{itemsep=.5ex, topsep=1ex, parsep=0pt, partopsep=0pt, leftmargin=3em, rightmargin=0em}
\usepackage[hang, flushmargin, multiple, bottom, stable]{footmisc}

\usepackage{fancyhdr}

\usepackage{natbib}
	\bibpunct[:]{(}{)}{,}{a}{}{,}
	\setlength{\bibsep}{1ex plus 0.3ex}
\renewcommand{\bibsection}{\part*{Bibliography}}

\usepackage{cite-ref-errors}

\setlength\parskip{.5\baselineskip}
\setlength\parindent{0pt}
\frenchspacing
\raggedbottom

\def\THIStitle{Embarking on PoLaR Explorations}
\def\THISsubtitle{A Framework for Intonational Annotation and Analysis}
\pagestyle{fancy}
	\fancyhead[LO]{\textit{\THIStitle}}
	\fancyhead[RO]{}
	\fancyhead[LE]{}
	\fancyhead[RE]{Ahn, Veilleux, Shattuck-Hufnagel, Brugos}
\pagestyle{plain}
\hypersetup{
	breaklinks=true,
	pdfauthor={Byron Ahn, Nanette Veilleux, Stefanie Shattuck-Hufnagel, and Alejna Brugos},
	pdftitle={\THIStitle: \THISsubtitle},
	bookmarks,
	bookmarksopen=true,
	colorlinks=false,
	allcolors=blue
}

\makeindex

\begin{document}
\frontmatter
\captionsetup{margin=1.5cm, skip=4pt, labelfont={bf, footnotesize}, textfont={footnotesize}}

\title{\THIStitle}
\author{Byron Ahn \and Nanette Veilleux \and Stefanie Shattuck-Hufnagel \and Alejna Brugos}
\date{\today}
%{\normalsize\textcolor{red}{These guidelines are still under active development.\\Find the latest version of the guidelines, as well as .wav files, .TextGrid files, and some scripts, at \url{https://www.polarlabels.com/}.}}

\begin{titlingpage}
\vspace*{\fill}
\begin{center}
\textbf{\huge{\THIStitle:\strut}}\\
\textbf{\huge{\THISsubtitle\strut}}
\\[6\baselineskip]
{\large{Byron Ahn,\strut\\Nanette Veilleux,\strut\\Stefanie Shattuck-Hufnagel,\strut\\and Alejna Brugos\strut}}\\[6\baselineskip]
{\large\textit{draft}}\\
November 2022
\end{center}
\vspace*{\fill}
\end{titlingpage}

\tableofcontents
\newpage
\listoffigures
\listoftables
\newpage

\mainmatter

\addchap{\lsPrefaceTitle} 
% \section*{Introduction}\label{sec:introduction}
%\addcontentsline{toc}{chapter}{Introduction}

This manuscript lays out an extensive set of guidelines that define the \textbf{PoLaR} framework for prosodic labelling, which is based on \textbf{Po}ints, \textbf{L}evels, \textbf{a}nd \textbf{R}anges.  The aim of this system is to provide a means of annotating (both \uline{separately} and \uline{more explicitly}) information such as f0 scaling and alignment, that is already taken into account by labellers using other intonational annotation systems, but in less explicit ways.

Before later chapters lay out the details of PoLaR, Chapter \ref{ch:background} provides some background on prosody, prosodic annotation, and the motivation for developing this framework. Here a reader can find a sketch of the basic properties of PoLaR —describing the fundamentals of the system and how it fosters the explicit labelling of both phonetic and phonological aspects of the prosody of an utterance.  These are transcribed on a number of different tiers, while relationships between these two types of labels can still be annotated. Chapter \ref{ch:background} also provides some motivating context, and ends with an overview of some advantages of PoLaR labelling. The reader who is primarily interested in learning how to label with PoLaR can skip ahead to Chapter \ref{ch:basics} (for PoLaR Basic labels) and Chapter \ref{ch:advanced} (for PoLaR Advanced labels), which serve as the annotation guidelines, i.e. as a tutorial on how to label using PoLaR.\footnote{Note: The guidelines found in this manuscript are for version 1.0 of PoLaR annotation (\citealt{ahn-21}). Any developments beyond version 1.0 will be available through the PoLaR repository: \href{https://doi.org/10.17605/OSF.IO/USBX5}{https://doi.org/10.17605/OSF.IO/USBX5}.}

After the official guidelines, Chapter \ref{ch:beyond} discusses ways in which PoLaR labelling is explicitly designed to be customized or expanded upon. This is followed by the final content chapter of this manuscript, Chapter \ref{ch:advantages}, which contains a discussion of some of the practical and theoretical advantages of PoLaR. Chapter \ref{ch:practical} concludes the manuscript with a reference guide to PoLaR, which is followed by published references and appendices.


\bibliographystyle{glossa.bst}
\bibliography{localbibliography.bib}

\end{document}	