%\addchap{\lsAbbreviationsTitle}
%% \addchap{Abbreviations and symbols}
%
%\begin{tabularx}{.45\textwidth}{lQ}
%... & \\
%... & \\
%\end{tabularx}
%\begin{tabularx}{.45\textwidth}{lQ}
%... & \\
%... & \\
%\end{tabularx}
%


%%%%%%%%%%%
%% FROM BYRON'S LATEX:
%%%%%%%%%%%
%\newglossaryentry{contour}
%{
%  name=contour,
%  description={\textit{see entries for ‘f0 contour’, ‘intonational contour’, and ‘prosodic contour’}}
%}
%
%\newglossaryentry{f0contour}
%{
%  name=f0 contour,
%  description={a description of changes in the f0 values in an utterance over time}
%%  ,
%%  parent=contour
%}
%
%\newglossaryentry{pitchcontour}
%{
%  name=pitch contour,
%  description={a description of changes in pitch over time (\textit{represents events in a listener’s mind})}
%%  ,
%%  parent=contour
%}
%
%\newglossaryentry{intonationalcontour}
%{
%  name=intonational contour,
%  description={an abstract sequence of pitch events over time (\textit{requires a grammar})}
%%  ,
%%  parent=contour
%}
%
%\newglossaryentry{fundamental frequency}
%{
%  name=fundamental frequency,
%  description={\textit{see entry for ‘f0’}}
%}
%
%\newglossaryentry{f0}
%{
%  name=f0,
%  description={fundamental frequency; a physical measure directly related to rate of vibration of the vocal folds, as reflected in the acoustic signal or articulatory measures}
%}
%\newglossaryentry{pitch}
%{
%  name=pitch,
%  description={an abstract psycho-perceptual phenomenon related to f0 (\textit{requires a listener with a mind})}
%}
%\newglossaryentry{intonation}
%{
%  name=intonation,
%  description={the arm of phonetics\slash phonology dealing with pitch patterns}
%}
%\newglossaryentry{prosody}
%{
%  name=prosody,
%  description={the arm of phonetics\slash phonology dealing with suprasegmental patterns, broadly speaking}
%}
%%\newglossaryentry{microprosody}
%%{
%%  name=microprosody,
%%  description={XXXX}
%%}
%
%
%%\textbf{[ COMING SOON ]}
%
%\glsaddall
%\glossarystyle{index}
%\printglossary[type=main]